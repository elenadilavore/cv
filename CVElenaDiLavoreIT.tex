% (c) 2002 Matthew Boedicker <mboedick@mboedick.org> (original author) http://mboedick.org
% (c) 2003-2007 David J. Grant <davidgrant-at-gmail.com> http://www.davidgrant.ca
% (c) 2008 Nathaniel Johnston <nathaniel@nathanieljohnston.com> http://www.nathanieljohnston.com
%
% (c) 2012 Scott Clark <sc932@cornell.edu> cam.cornell.edu/~sc932
%
%This work is licensed under the Creative Commons Attribution-Noncommercial-Share Alike 2.5 License. To view a copy of this license, visit http://creativecommons.org/licenses/by-nc-sa/2.5/ or send a letter to Creative Commons, 543 Howard Street, 5th Floor, San Francisco, California, 94105, USA.

\documentclass[letterpaper,11pt]{article}
\newlength{\outerbordwidth}
\pagestyle{empty}
\raggedbottom
\raggedright
\usepackage[svgnames]{xcolor}
\usepackage{framed}
\usepackage{tocloft}
\usepackage{array}
\usepackage{hyperref}
\usepackage[none]{hyphenat}
\definecolor{bordeaux}{HTML}{4b1121}
\hypersetup{colorlinks=true, linkcolor=bordeaux, citecolor=bordeaux, urlcolor=bordeaux}
\usepackage[sorting = ydnt, maxbibnames=9]{biblatex}
\nocite{*}
{\def\section*#1{}\bibliography{publicationsbib}}
%-----------------------------------------------------------
%Edit these values as you see fit
\usepackage[italian]{babel}

\setlength{\outerbordwidth}{3pt}  % Width of border outside of title bars
\definecolor{shadecolor}{gray}{0.75}  % Outer background color of title bars (0 = black, 1 = white)
\definecolor{shadecolorB}{gray}{0.93}  % Inner background color of title bars


%-----------------------------------------------------------
%Margin setup

\setlength{\evensidemargin}{-0.25in}
\setlength{\headheight}{0in}
\setlength{\headsep}{0in}
\setlength{\oddsidemargin}{-0.25in}
\setlength{\paperheight}{11in}
\setlength{\paperwidth}{8.5in}
\setlength{\tabcolsep}{0in}
\setlength{\textheight}{9.5in}
\setlength{\textwidth}{7in}
\setlength{\topmargin}{-0.3in}
\setlength{\topskip}{0in}
\setlength{\voffset}{0.1in}


%-----------------------------------------------------------
%Custom commands
\newcommand{\resitem}[1]{\item #1 \vspace{-2pt}}
\newcommand{\resheading}[1]{\vspace{8pt}
  \parbox{\textwidth}{\setlength{\FrameSep}{\outerbordwidth}
    \begin{shaded}
\setlength{\fboxsep}{0pt}\framebox[\textwidth][l]{\setlength{\fboxsep}{4pt}\fcolorbox{shadecolorB}{shadecolorB}{\textbf{\sffamily{\mbox{~}\makebox[6.762in][l]{\large #1} \vphantom{p\^{E}}}}}}
    \end{shaded}
  }\vspace{-5pt}
}
\newcommand{\ressubheading}[4]{
\begin{tabular*}{6.5in}{l@{\cftdotfill{\cftsecdotsep}\extracolsep{\fill}}r}
		\textbf{#1} & #2 \\
		\textit{#3} & \textit{#4} \\
\end{tabular*}\vspace{-6pt}}
%-----------------------------------------------------------


\begin{document}

\begin{tabular*}{7in}{l@{\extracolsep{\fill}}r}
\textbf{\Large Elena Di Lavore} & \textbf{\today} \\
Nata a Milano, Italia, 13/08/1995 & elena.dilavore@di.unipi.it \\
Indirizzo: Largo Bruno Pontecorvo 3, Pisa, Italia & pagina web: \href{https://elenadilavore.github.io/}{https://elenadilavore.github.io/} \\
\end{tabular*}
\\
\ \\
\ \\
Assegnista al dipartimento di Informatica all'Università di Pisa.

%%%%%%%%%%%%%%%%%%%%%%%%%%%%%%
\resheading{Riconoscimenti}
%%%%%%%%%%%%%%%%%%%%%%%%%%%%%%
\vspace{-2pt}
\begin{center}
  \begin{tabular*}{6.6in}{l@{\extracolsep{\fill}}r}
	\multicolumn{2}{c}{Grant europea (EuroProofNet) per una Short Term Scientific Mission\cftdotfill{\cftdotsep} 2024}\\
	\multicolumn{2}{c}{Distinguished paper~\cite{w2023partialmarkov} alla conferenza ACT\cftdotfill{\cftdotsep} 2023}\\
	\multicolumn{2}{c}{Distinguished paper~\cite{w2022monoidalstreams} alla conferenza ACT\cftdotfill{\cftdotsep} 2022}\\
	\multicolumn{2}{c}{Kleene Award per il migliore articolo di soli studenti~\cite{2022monoidalstreams}, ACM/IEEE LiCS\cftdotfill{\cftdotsep} 2022}\\
	\multicolumn{2}{c}{Distinguished paper~\cite{2022monoidalstreams} alla conferenza ACM/IEEE LiCS\cftdotfill{\cftdotsep} 2023}\\
	\multicolumn{2}{c}{Esonero per merito (Politecnico di Milano) \cftdotfill{\cftdotsep} 2016--2017}\\
	\multicolumn{2}{c}{Esonero per merito (Politecnico di Milano) \cftdotfill{\cftdotsep} 2015--2016}\\
	\multicolumn{2}{c}{Premio per le migliori matricole (Politecnico di Milano) \cftdotfill{\cftdotsep} 2015}\\
	\vphantom{E}
  \end{tabular*}
\end{center}\vspace*{-16pt}
%%%%%%%%%%%%%%%%%%%%%%%%%%%%%%
\resheading{Istruzione}
%%%%%%%%%%%%%%%%%%%%%%%%%%%%%%
\begin{itemize}

\item[] \ressubheading{Tallinna Tehnika{\"u}likool}{Estonia}{Dottorato in Informatica}{2019 - 2023}

\begin{itemize}
	\resitem{Titolo della tesi: Monoidal Width \\
		Relatore: Professor Pawe{\l} Soboci{\'n}ski\\
		Controrelatori: Professor Samson Abramski e professor Dan Marsden}

\end{itemize}

\item[] \ressubheading{University of Oxford}{Inghilterra}{MSc in Mathematics and Foundations of Computer Science}{2018 - 2019}

\begin{itemize}
	\resitem{Titolo della tesi: Subgame Perfection in Compositional Game Theory \\
	Relatori: Dr Jules Hedges, professor Jamie Vicary}
	\resitem{Voto: merit}
\end{itemize}

\item[] \ressubheading{Universit\`a di Pisa}{Italia}{Laurea Triennale in Matematica}{2017 - 2018}

\begin{itemize}
	\resitem{Titolo della tesi: Data-driven Estimation for Nash Equilibria \\
	Relatore: Professor Giancarlo Bigi}
	\resitem{Voto: 110 e lode laude}
\end{itemize}

\item[] \ressubheading{Politecnico di Milano}{Italia}{Laurea Triennale in Ingegneria Matematica}{2014 - 2017}

\begin{itemize}
	\resitem{Titolo della tesi: Floquet Theory Applied to a Perturbed Wave Equation \\
	Relatore: Professor Gianni Arioli}
	\resitem{Voto: 110 cum laude / 110}
	\resitem{Studi all'estero: programma Erasmus a Linnaeus University, V\"axj\"o, Svezia}
\end{itemize}
\end{itemize}
\newpage
%%%%%%%%%%%%%%%%%%%%%%%%%%%%%%
\resheading{Incarichi accademici}
%%%%%%%%%%%%%%%%%%%%%%%%%%%%%%
\begin{itemize}
  \item (da dicembre 2023) Membro dello steering committee della Adjoint School.
  \item (da maggio 2023) Membro dell'executive board del giornale Compositionality.
  \item (settembre 2022) Organizzatore locale del nono Symposium on Compositional Structures.
  \item (maggio 2022) Membro del program committee della conferenza Applied Category Theory.
  \item (2021-2023) Organizzatore della Adjoint School.
  \item Reviewer per conferenze (LiCS, MFPS, ...) e giornali (TAC, RAIRO, MSCS, ...).
\end{itemize}

%%%%%%%%%%%%%%%%%%%%%%%%%%%%%%
\resheading{Insegnamento}
%%%%%%%%%%%%%%%%%%%%%%%%%%%%%%
\begin{itemize}
  \item (secondo semestre 2023) Docente per il corso Introduction to Category Theory.
  \item (secondo semestre 2022) Esercitatrice per il corso Introduction to Category Theory.
  \item (secondo semestre 2021) Esercitatrice per il corso Introduction to Category Theory.
\end{itemize}

%%%%%%%%%%%%%%%%%%%%%%%%%%%%%%
\resheading{Scuole di dottorato e visite accademiche}
%%%%%%%%%%%%%%%%%%%%%%%%%%%%%%
\begin{itemize}
  \item (novembre 2022) Visita accademica a Jamie Vicary, Computer Laboratory, Cambridge, UK.
  \item (luglio 2022) Visita accademica a Filippo Bonchi e Fabio Gadducci, Dipartimento di Informatica, Pisa, Italia.
  \item (2019-2020) Studentessa della Adjoint School sotto la supervisione di Valeria de Paiva.
  \item (marzo 2020) Studentessa alla Estonian Winter School in Computer Science.
\end{itemize}

%%%%%%%%%%%%%%%%%%%%%%%%%%%%%%
\resheading{Presentazioni accademiche}
%%%%%%%%%%%%%%%%%%%%%%%%%%%%%%
\begin{itemize}
  \item (08/03/2024) Oxford Advanced Seminar on Informatic Structures, University of Oxford, UK.
  \item (08/01/2024) Directions and Perspectives in the \(\lambda\)-calculus, Università di Bologna, Italia.
  \item (22/11/2023) 34th Nordic Workshop on Programming Theory, Mälardalens universitet, Svezia.
  \item (01/08/2023) Conferenza Applied Category Theory, University of Maryland, USA.
  \item (18/07/2023) Coresources workshop, University of Cambridge, UK.
  \item (17/07/2023) Bob Walters Memorial, Tallinn, Estonia.
  \item (27/06/2023) Conferenza Logic in Computer Science (LiCS), Boston University, USA.
  \item (25/06/2023) International Workshop on Quantitative Logical methods, Boston University, USA.
  \item (14/06/2023) Categories Networking Project opening workshop, University of Edinburgh, UK.
  \item (26/04/2023) Italian Category Theory Fest 2023, online.
  \item (09/03/2023) Tallinn Categories Seminar, Tallinn, Estonia.
  \item (10/11/2022) Seminario al Cambridge Logical Structures Hub, Cambridge, UK.
  \item (01/08/2022) Women in Logic workshop, Technion, Israele.
  \item (22/07/2022) Conferenza Applied Category Theory, University of Strathclyde, UK.
  \item (08/07/2022) Seminario del Dipartimento di Informatica, Pisa, Italia.
  \item (26/06/2022) 29th Foundational Methods in Computer Science, University of Calgary, Canada.
  \item (25/05/2022) Comonads Meetup, University of Cambridge, UK.
  \item (24/05/2022) Mathematical Foundations Seminar, University of Bath, UK.
  \item (13/12/2021) Symposium on Compositional Structures 8, Tallinna Tehnika{\"u}likool, Estonia.
  \item (26/01/2021) Conferenza Computer Science Logic 2021, Univerza v Ljubljani, Slovenia.
\end{itemize}

%%%%%%%%%%%%%%%%%%%%%%%%%%%%%%
\resheading{Altro}
%%%%%%%%%%%%%%%%%%%%%%%%%%%%%%

\begin{itemize}
  \item[] \textbf{Lingue:} italiano (lingua madre), inglese (C1).
  \item[] \textbf{Linguaggi di programmazione:} conoscenze di base di Idris, Matlab, C, R.
\end{itemize}

%%%%%%%%%%%%%%%%%%%%%%%%%%%%%%
\resheading{Pubblicazioni scientifiche}
%%%%%%%%%%%%%%%%%%%%%%%%%%%%%%
%\printbibliography[heading=none]
\newrefcontext[sorting=ydnt]
\printbibliography[keyword={pub}, heading={none}]

% \textbf{Note.} As customary in mathematics, all my publications list the authors in alphabetical order.

%%%%%%%%%%%%%%%%%%%%%%%%%%%%%%
\resheading{Extended abstracts in workshops (peer-reviewed)}
%%%%%%%%%%%%%%%%%%%%%%%%%%%%%%
%\printbibliography[heading=none]
\newrefcontext[sorting=ydnt]
\printbibliography[keyword={abs}, heading={none}]

\end{document}
