% (c) 2002 Matthew Boedicker <mboedick@mboedick.org> (original author) http://mboedick.org
% (c) 2003-2007 David J. Grant <davidgrant-at-gmail.com> http://www.davidgrant.ca
% (c) 2008 Nathaniel Johnston <nathaniel@nathanieljohnston.com> http://www.nathanieljohnston.com
%
% (c) 2012 Scott Clark <sc932@cornell.edu> cam.cornell.edu/~sc932
%
%This work is licensed under the Creative Commons Attribution-Noncommercial-Share Alike 2.5 License. To view a copy of this license, visit http://creativecommons.org/licenses/by-nc-sa/2.5/ or send a letter to Creative Commons, 543 Howard Street, 5th Floor, San Francisco, California, 94105, USA.

\documentclass[letterpaper,11pt]{article}
\newlength{\outerbordwidth}
\pagestyle{empty}
\raggedbottom
\raggedright
\usepackage[svgnames]{xcolor}
\usepackage{framed}
\usepackage{tocloft}
\usepackage{array}
\usepackage{hyperref}
\usepackage[none]{hyphenat}
\definecolor{bordeaux}{HTML}{4b1121}
\hypersetup{colorlinks=true, linkcolor=bordeaux, citecolor=bordeaux, urlcolor=bordeaux}
\usepackage[sorting = ydnt, maxbibnames=9]{biblatex}
\nocite{*}
{\def\section*#1{}\bibliography{publicationsbib}}
%-----------------------------------------------------------
%Edit these values as you see fit

\setlength{\outerbordwidth}{3pt}  % Width of border outside of title bars
\definecolor{shadecolor}{gray}{0.75}  % Outer background color of title bars (0 = black, 1 = white)
\definecolor{shadecolorB}{gray}{0.93}  % Inner background color of title bars


%-----------------------------------------------------------
%Margin setup

\setlength{\evensidemargin}{-0.25in}
\setlength{\headheight}{0in}
\setlength{\headsep}{0in}
\setlength{\oddsidemargin}{-0.25in}
\setlength{\paperheight}{11in}
\setlength{\paperwidth}{8.5in}
\setlength{\tabcolsep}{0in}
\setlength{\textheight}{9.5in}
\setlength{\textwidth}{7in}
\setlength{\topmargin}{-0.3in}
\setlength{\topskip}{0in}
\setlength{\voffset}{0.1in}


%-----------------------------------------------------------
%Custom commands
\newcommand{\resitem}[1]{\item #1 \vspace{-2pt}}
\newcommand{\resheading}[1]{\vspace{8pt}
  \parbox{\textwidth}{\setlength{\FrameSep}{\outerbordwidth}
    \begin{shaded}
\setlength{\fboxsep}{0pt}\framebox[\textwidth][l]{\setlength{\fboxsep}{4pt}\fcolorbox{shadecolorB}{shadecolorB}{\textbf{\sffamily{\mbox{~}\makebox[6.762in][l]{\large #1} \vphantom{p\^{E}}}}}}
    \end{shaded}
  }\vspace{-5pt}
}
\newcommand{\ressubheading}[4]{
\begin{tabular*}{6.5in}{l@{\cftdotfill{\cftsecdotsep}\extracolsep{\fill}}r}
		\textbf{#1} & #2 \\
		\textit{#3} & \textit{#4} \\
\end{tabular*}\vspace{-6pt}}
%-----------------------------------------------------------


\begin{document}

\begin{tabular*}{7in}{l@{\extracolsep{\fill}}r}
\textbf{\Large Elena Di Lavore} & \textbf{\today} \\
Born in Milano, Italy, 13/08/1995 & elenatalita@gmail.com \\
Address: Akadeemia tee 21-1, Tallinn 12611, Estonia & +393665334335 \\
\end{tabular*}
\\


%%%%%%%%%%%%%%%%%%%%%%%%%%%%%%
\resheading{Education}
%%%%%%%%%%%%%%%%%%%%%%%%%%%%%%
\begin{itemize}

\item[] \ressubheading{Tallinn University of Technology}{Estonia}{PhD}{2019 - present}

\begin{itemize}
	\resitem{Thesis topic: Monoidal Width \\
	Supervisor: Professor Pawe{\l} Soboci{\'n}ski}
	\resitem{Teaching experience as TA for the introductory course on Category Theory}

\end{itemize}

\item[] \ressubheading{University of Oxford}{United Kingdom}{MSc in Mathematics and Foundations of Computer Science}{2018 - 2019}

\begin{itemize}
	\resitem{Thesis: Subgame Perfection in Compositional Game Theory \\
	Supervisors: Dr Jules Hedges, Dr Jamie Vicary}
	\resitem{Mark: merit}
	\resitem{Courses:\\ \begin{tabular}{p{0.4 \textwidth}@{\hskip 2em} p{0.4 \textwidth}}
	Categories, proofs and processes & Quantum computer science \\
	Categorical quantum mechanics & Graph theory \\
	Axiomatic set theory & Lambda calculus and types	\\
	Computational number theory
	\end{tabular}}
\end{itemize}

\item[] \ressubheading{Universit\`a di Pisa}{Italy}{BSc in Mathematics}{2017 - 2018}

\begin{itemize}
	\resitem{Thesis: Data-driven Estimation for Nash Equilibria \\
	Supervisor: Professor Giancarlo Bigi}
	\resitem{Mark: 110 cum laude / 110}
	\resitem{Courses:\\ \begin{tabular}{p{0.4 \textwidth}@{\hskip 2em} p{0.4 \textwidth}}
	Arithmetic & Elements of set theory \\
	Experimental laboratory of computational mathematics & Physics 3 \\
	Geometry 2 & Algebra 1 \\
	Mathematical logic & Game theory
	
	\end{tabular}}
\end{itemize}

\item[] \ressubheading{Politecnico di Milano}{Italy}{BSc in Mathematical Engineering}{2014 - 2017}

\begin{itemize}
	\resitem{Thesis: Floquet Theory Applied to a Perturbed Wave Equation \\
	Supervisor: Professor Gianni Arioli}
	\resitem{Mark: 110 cum laude / 110}
	\resitem{Studies abroad: Erasmus program at Linnaeus University, V\"axj\"o, Sweden (August 2016 - January 2017)}
	\resitem{Courses:\\ \begin{tabular}{p{0.4 \textwidth}@{\hskip 2em} p{0.4 \textwidth}}
	Computer science & Mathematical Analysis 1, 2, 3 \\
	Chemistry & Experimental physics 1, 2 \\
	Business economics and organisation & Statistics \\
	Electrical engineering & Linear algebra and geometry\\
	Fundamentals of automatic controls & Probability \\
	Mathematical physics & Numerical mathematics\\
	Advanced analog electronics & P.D.E. analytical and numerical methods \\
	Algorithms and advanced data structures & Functional analysis \\
	Analysis of structures & Models and methods for statistical inference \\
	Algebraic structures 1
	\end{tabular}	 }
\end{itemize}
\end{itemize}

%%%%%%%%%%%%%%%%%%%%%%%%%%%%%%
\resheading{Publications}
%%%%%%%%%%%%%%%%%%%%%%%%%%%%%%
\textbf{Note.} As customary in mathematics, all my publications list the authors in alphabetical order.
\printbibliography[heading=none]
%%%%%%%%%%%%%%%%%%%%%%%%%%%%%%
\resheading{Academic commitments}
%%%%%%%%%%%%%%%%%%%%%%%%%%%%%%
\begin{itemize}
  \item (since May 2023) Member of the executive board of the journal Compositionality.
  \item (September 2022) Local co-organiser of the 9\(^{th}\) Symposium on Compositional Structures.
  \item (May 2022) Program committee member of the Applied Category Theory conference.
  \item (2022-2023) Organiser of the Applied Category Theory Adjoint School.
  \item Reviewer for conferences (LiCS, MFPS, ACT, ...) and journals (TAC, RAIRO, Compositionality, ...).
\end{itemize}
%%%%%%%%%%%%%%%%%%%%%%%%%%%%%%
\resheading{Other skills}
%%%%%%%%%%%%%%%%%%%%%%%%%%%%%%

\begin{itemize}
  \item[] \textbf{Language skills}
	\begin{itemize}
	  \item Italian: mother tongue
	  \item English: C1
	\end{itemize}
  \item[] \textbf{Programming languages}
	\begin{itemize}
	  \item Idris, Matlab
	  \item C, Java: basic knowledge
	  \item R, SQL: basic knowledge
	\end{itemize}
\end{itemize}

%%%%%%%%%%%%%%%%%%%%%%%%%%%%%%
\resheading{Awards}
%%%%%%%%%%%%%%%%%%%%%%%%%%%%%%
\vspace{-2pt}
\begin{center}\begin{tabular*}{6.6in}{l@{\extracolsep{\fill}}r}
				\multicolumn{2}{c}{Kleene Award for~\cite{2022monoidalstreams}\cftdotfill{\cftdotsep} 2022}\\
				\multicolumn{2}{c}{Exemptions for High Academic Performance (Politecnico di Milano) \cftdotfill{\cftdotsep} 2015--2017}\\
				\multicolumn{2}{c}{Best Freshers Award (Politecnico di Milano) \cftdotfill{\cftdotsep} 2015}\\
				\vphantom{E}
			  \end{tabular*}
			\end{center}\vspace*{-16pt}


\end{document}
