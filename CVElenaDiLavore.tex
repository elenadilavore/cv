% (c) 2002 Matthew Boedicker <mboedick@mboedick.org> (original author) http://mboedick.org
% (c) 2003-2007 David J. Grant <davidgrant-at-gmail.com> http://www.davidgrant.ca
% (c) 2008 Nathaniel Johnston <nathaniel@nathanieljohnston.com> http://www.nathanieljohnston.com
%
% (c) 2012 Scott Clark <sc932@cornell.edu> cam.cornell.edu/~sc932
%
%This work is licensed under the Creative Commons Attribution-Noncommercial-Share Alike 2.5 License. To view a copy of this license, visit http://creativecommons.org/licenses/by-nc-sa/2.5/ or send a letter to Creative Commons, 543 Howard Street, 5th Floor, San Francisco, California, 94105, USA.

\documentclass[letterpaper,11pt]{article}
\newlength{\outerbordwidth}
\pagestyle{empty}
\raggedbottom
\raggedright
\usepackage[svgnames]{xcolor}
\usepackage{framed}
\usepackage{tocloft}
\usepackage{array}
\usepackage[none]{hyphenat}


%-----------------------------------------------------------
%Edit these values as you see fit

\setlength{\outerbordwidth}{3pt}  % Width of border outside of title bars
\definecolor{shadecolor}{gray}{0.75}  % Outer background color of title bars (0 = black, 1 = white)
\definecolor{shadecolorB}{gray}{0.93}  % Inner background color of title bars


%-----------------------------------------------------------
%Margin setup

\setlength{\evensidemargin}{-0.25in}
\setlength{\headheight}{0in}
\setlength{\headsep}{0in}
\setlength{\oddsidemargin}{-0.25in}
\setlength{\paperheight}{11in}
\setlength{\paperwidth}{8.5in}
\setlength{\tabcolsep}{0in}
\setlength{\textheight}{9.5in}
\setlength{\textwidth}{7in}
\setlength{\topmargin}{-0.3in}
\setlength{\topskip}{0in}
\setlength{\voffset}{0.1in}


%-----------------------------------------------------------
%Custom commands
\newcommand{\resitem}[1]{\item #1 \vspace{-2pt}}
\newcommand{\resheading}[1]{\vspace{8pt}
  \parbox{\textwidth}{\setlength{\FrameSep}{\outerbordwidth}
    \begin{shaded}
\setlength{\fboxsep}{0pt}\framebox[\textwidth][l]{\setlength{\fboxsep}{4pt}\fcolorbox{shadecolorB}{shadecolorB}{\textbf{\sffamily{\mbox{~}\makebox[6.762in][l]{\large #1} \vphantom{p\^{E}}}}}}
    \end{shaded}
  }\vspace{-5pt}
}
\newcommand{\ressubheading}[4]{
\begin{tabular*}{6.5in}{l@{\cftdotfill{\cftsecdotsep}\extracolsep{\fill}}r}
		\textbf{#1} & #2 \\
		\textit{#3} & \textit{#4} \\
\end{tabular*}\vspace{-6pt}}
%-----------------------------------------------------------


\begin{document}

\begin{tabular*}{7in}{l@{\extracolsep{\fill}}r}
\textbf{\Large Elena Di Lavore} & \textbf{\today} \\
Born in Milano, Italy, 13/08/1995 & elenatalita@gmail.com \\
Address: Akadeemia tee 10, Tallinn 12611, Estonia & +393665334335 \\
\end{tabular*}
\\


%%%%%%%%%%%%%%%%%%%%%%%%%%%%%%
\resheading{Education}
%%%%%%%%%%%%%%%%%%%%%%%%%%%%%%
\begin{itemize}

\item[] \ressubheading{Tallinn University of Technology}{Estonia}{PhD}{2019 - present}

\begin{itemize}
	\resitem{Thesis topic: Compositional game theory \\
	Supervisors: Dr Julian Hedges, Professor Pawel Sobocinski}
\end{itemize}

\item[] \ressubheading{University of Oxford}{United Kingdom}{MSc in Mathematics and Foundations of Computer Science}{2018 - 2019}

\begin{itemize}
	\resitem{Thesis: Subgame perfection in compositional game theory \\
	Supervisors: Dr Julian Hedges, Dr Jamie Vicary}
	\resitem{Mark: merit}
	\resitem{Courses:\\ \begin{tabular}{p{0.4 \textwidth}@{\hskip 2em} p{0.4 \textwidth}}
	Categories, proofs and processes & Quantum computer science \\
	Categorical quantum mechanics & Graph theory \\
	Axiomatic set theory & Lambda calculus and types	\\
	Computational number theory
	\end{tabular}}
\end{itemize}

\item[] \ressubheading{Universit\`a di Pisa}{Italy}{BSc in Mathematics}{2017 - 2018}

\begin{itemize}
	\resitem{Thesis: Data-driven Estimation for Nash Equilibria \\
	Supervisor: Professor Giancarlo Bigi}
	\resitem{Mark: 110 cum laude / 110}
	\resitem{Courses:\\ \begin{tabular}{p{0.4 \textwidth}@{\hskip 2em} p{0.4 \textwidth}}
	Arithmetic & Elements of set theory \\
	Experimental laboratory of computational mathematics & Physics 3 \\
	Geometry 2 & Algebra 1 \\
	Mathematical logic & Game theory
	
	\end{tabular}}
\end{itemize}

\item[] \ressubheading{Politecnico di Milano}{Italy}{BSc in Mathematical Engineering}{2014 - 2017}

\begin{itemize}
	\resitem{Thesis: Floquet Theory Applied to a Perturbed Wave Equation \\
	Supervisor: Professor Gianni Arioli}
	\resitem{Mark: 110 cum laude / 110}
	\resitem{Studies abroad: Erasmus program at Linnaeus University, V\"axj\"o, Sweden (August 2016 - January 2017)}
	\resitem{Courses:\\ \begin{tabular}{p{0.4 \textwidth}@{\hskip 2em} p{0.4 \textwidth}}
	Computer science & Mathematical Analysys 1, 2, 3 \\
	Chemistry & Experimental physics 1, 2 \\
	Business economics and organization & Statistics \\
	Electrical engineering & Linear algebra and geometry\\
	Fundamentals of automatic controls & Probability \\
	Mathematical physics & Numerical mathematics\\
	Advanced analog electronics & P.D.E. analytical and numerical methods \\
	Algorithms and advanced data structures & Functional analysis \\
	Analysis of structures & Models and methods for statistical inference \\
	Algebraic structures 1
	\end{tabular}	 }
\end{itemize}

\end{itemize}


%%%%%%%%%%%%%%%%%%%%%%%%%%%%%%
\resheading{Interests and other skills}
%%%%%%%%%%%%%%%%%%%%%%%%%%%%%%

\begin{itemize}
 \item[] \textbf{Academic interests}
 \begin{itemize}
  \item Applied category theory, Compositional game theory
  \item Attended the conferences SYCO2 (Glasgow, December 2018), SYCO3 (Oxford, March 2019), CT 2019 (Edinburgh, July 2019), ACT (July 2019), STRINGS3 and SYCO5 (Birmingham, September 2019), ItaCa (Milan, December 2019)
 \end{itemize}
 \item[] \textbf{Language skills}
 \begin{itemize}
  \item Italian: mother tongue
  \item English: C1
 \end{itemize}
 \item[] \textbf{Programming languages}
 \begin{itemize}
  \item C, Java : basic knowledge
  \item MATLAB, R, SQL : basic knowledge
 \end{itemize}
\end{itemize}

%%%%%%%%%%%%%%%%%%%%%%%%%%%%%%
\resheading{Grants}
%%%%%%%%%%%%%%%%%%%%%%%%%%%%%%
	\vspace{-2pt}
	\begin{center}\begin{tabular*}{6.6in}{l@{\extracolsep{\fill}}r}
		\multicolumn{2}{c}{Best Freshers Award (Politecnico di Milano) \cftdotfill{\cftdotsep} 2015}\\
        \multicolumn{2}{c}{Exemptions for High Academic Performance (Politecnico di Milano) \cftdotfill{\cftdotsep} 2015-2017}\\
		\vphantom{E}
\end{tabular*}
\end{center}\vspace*{-16pt}


\end{document}
