% (c) 2002 Matthew Boedicker <mboedick@mboedick.org> (original author) http://mboedick.org
% (c) 2003-2007 David J. Grant <davidgrant-at-gmail.com> http://www.davidgrant.ca
% (c) 2008 Nathaniel Johnston <nathaniel@nathanieljohnston.com> http://www.nathanieljohnston.com
%
% (c) 2012 Scott Clark <sc932@cornell.edu> cam.cornell.edu/~sc932
%
%This work is licensed under the Creative Commons Attribution-Noncommercial-Share Alike 2.5 License. To view a copy of this license, visit http://creativecommons.org/licenses/by-nc-sa/2.5/ or send a letter to Creative Commons, 543 Howard Street, 5th Floor, San Francisco, California, 94105, USA.

\documentclass[letterpaper,11pt]{article}
\newlength{\outerbordwidth}
\pagestyle{empty}
\raggedbottom
\raggedright
\usepackage[svgnames]{xcolor}
\usepackage{framed}
\usepackage{tocloft}
\usepackage{array}
\usepackage{hyperref}
\usepackage[none]{hyphenat}
\definecolor{bordeaux}{HTML}{4b1121}
\hypersetup{colorlinks=true, linkcolor=bordeaux, citecolor=bordeaux, urlcolor=bordeaux}
\usepackage[sorting = ydnt, maxbibnames=9]{biblatex}
\nocite{*}
{\def\section*#1{}\bibliography{publicationsbib}}
%-----------------------------------------------------------
%Edit these values as you see fit

\setlength{\outerbordwidth}{3pt}  % Width of border outside of title bars
\definecolor{shadecolor}{gray}{0.75}  % Outer background color of title bars (0 = black, 1 = white)
\definecolor{shadecolorB}{gray}{0.93}  % Inner background color of title bars


%-----------------------------------------------------------
%Margin setup

\setlength{\evensidemargin}{-0.25in}
\setlength{\headheight}{0in}
\setlength{\headsep}{0in}
\setlength{\oddsidemargin}{-0.25in}
\setlength{\paperheight}{11in}
\setlength{\paperwidth}{8.5in}
\setlength{\tabcolsep}{0in}
\setlength{\textheight}{9.5in}
\setlength{\textwidth}{7in}
\setlength{\topmargin}{-0.3in}
\setlength{\topskip}{0in}
\setlength{\voffset}{0.1in}


%-----------------------------------------------------------
%Custom commands
\newcommand{\resitem}[1]{\item #1 \vspace{-2pt}}
\newcommand{\resheading}[1]{\vspace{8pt}
  \parbox{\textwidth}{\setlength{\FrameSep}{\outerbordwidth}
    \begin{shaded}
\setlength{\fboxsep}{0pt}\framebox[\textwidth][l]{\setlength{\fboxsep}{4pt}\fcolorbox{shadecolorB}{shadecolorB}{\textbf{\sffamily{\mbox{~}\makebox[6.762in][l]{\large #1} \vphantom{p\^{E}}}}}}
    \end{shaded}
  }\vspace{-5pt}
}
\newcommand{\ressubheading}[4]{
\begin{tabular*}{6.5in}{l@{\cftdotfill{\cftsecdotsep}\extracolsep{\fill}}r}
		\textbf{#1} & #2 \\
		\textit{#3} & \textit{#4} \\
\end{tabular*}\vspace{-6pt}}
%-----------------------------------------------------------


\begin{document}

\begin{tabular*}{7in}{l@{\extracolsep{\fill}}r}
\textbf{\Large Elena Di Lavore} & \textbf{\today} \\
Born in Milano, Italy, 13/08/1995 & elena.dilavore@di.unipi.it \\
Address: Largo Bruno Pontecorvo 3, Pisa, Italy & personal webpage: \href{https://elenadilavore.github.io/}{https://elenadilavore.github.io/} \\
\end{tabular*}
\\
\ \\
\ \\
Postdoctoral researcher in theoretical computer science at the University of Pisa.

%%%%%%%%%%%%%%%%%%%%%%%%%%%%%%
\resheading{Awards}
%%%%%%%%%%%%%%%%%%%%%%%%%%%%%%
\vspace{-2pt}
\begin{center}\begin{tabular*}{6.6in}{l@{\extracolsep{\fill}}r}
				\multicolumn{2}{c}{European (EuroProofNet) grant for a Short Term Scientific Mission\cftdotfill{\cftdotsep} 2024}\\
				\multicolumn{2}{c}{Kleene Award to the best student paper~\cite{2022monoidalstreams}, ACM/IEEE LiCS\cftdotfill{\cftdotsep} 2022}\\
				\multicolumn{2}{c}{Exemptions for High Academic Performance (Politecnico di Milano) \cftdotfill{\cftdotsep} 2015--2017}\\
				\multicolumn{2}{c}{Best Freshers Award (Politecnico di Milano) \cftdotfill{\cftdotsep} 2015}\\
				\vphantom{E}
			  \end{tabular*}
			\end{center}\vspace*{-16pt}
%%%%%%%%%%%%%%%%%%%%%%%%%%%%%%
\resheading{Education}
%%%%%%%%%%%%%%%%%%%%%%%%%%%%%%
\begin{itemize}

\item[] \ressubheading{Tallinn University of Technology}{Estonia}{PhD in Theoretical Computer Science}{2019 - 2023}

\begin{itemize}
	\resitem{Thesis: Monoidal Width \\
	Supervisor: Professor Pawe{\l} Soboci{\'n}ski}

\end{itemize}

\item[] \ressubheading{University of Oxford}{United Kingdom}{MSc in Mathematics and Foundations of Computer Science}{2018 - 2019}

\begin{itemize}
	\resitem{Thesis: Subgame Perfection in Compositional Game Theory \\
	Supervisors: Dr Jules Hedges, Dr Jamie Vicary}
	\resitem{Mark: merit}
\end{itemize}

\item[] \ressubheading{Universit\`a di Pisa}{Italy}{BSc in Mathematics}{2017 - 2018}

\begin{itemize}
	\resitem{Thesis: Data-driven Estimation for Nash Equilibria \\
	Supervisor: Professor Giancarlo Bigi}
	\resitem{Mark: 110 cum laude / 110}
\end{itemize}

\item[] \ressubheading{Politecnico di Milano}{Italy}{BSc in Mathematical Engineering}{2014 - 2017}

\begin{itemize}
	\resitem{Thesis: Floquet Theory Applied to a Perturbed Wave Equation \\
	Supervisor: Professor Gianni Arioli}
	\resitem{Mark: 110 cum laude / 110}
	\resitem{Studies abroad: Erasmus program at Linnaeus University, V\"axj\"o, Sweden}
\end{itemize}
\end{itemize}
\newpage
%%%%%%%%%%%%%%%%%%%%%%%%%%%%%%
\resheading{Academic commitments}
%%%%%%%%%%%%%%%%%%%%%%%%%%%%%%
\begin{itemize}
  \item (since Dec 2023) Member of the steering committee of the Adjoint School.
  \item (since May 2023) Member of the executive board of the journal Compositionality.
  \item (Sep 2022) Local co-organiser of the 9\(^{th}\) Symposium on Compositional Structures.
  \item (May 2022) Program committee member of the Applied Category Theory conference.
  \item (2021-2023) Organiser of the Adjoint School.
  \item Reviewer for conferences (LiCS, MFPS, ...) and journals (TAC, RAIRO, MSCS, ...).
\end{itemize}

%%%%%%%%%%%%%%%%%%%%%%%%%%%%%%
\resheading{Teaching experience}
%%%%%%%%%%%%%%%%%%%%%%%%%%%%%%
\begin{itemize}
  \item (Spring 2023) Lecturer in the Introduction to Category Theory course.
  \item (Spring 2022) Teaching assistant in the Introduction to Category Theory course.
  \item (Spring 2021) Teaching assistant in the Introduction to Category Theory course.
\end{itemize}

%%%%%%%%%%%%%%%%%%%%%%%%%%%%%%
\resheading{Doctoral schools}
%%%%%%%%%%%%%%%%%%%%%%%%%%%%%%
\begin{itemize}
  \item (2019-2020) Student in the Adjoint School mentored by Valeria de Paiva.
  \item (Mar 2020) Estonian Winter School in Computer Science
\end{itemize}

%%%%%%%%%%%%%%%%%%%%%%%%%%%%%%
\resheading{Academic talks}
%%%%%%%%%%%%%%%%%%%%%%%%%%%%%%
\begin{itemize}
  \item (08 Mar 2024) Oxford Advanced Seminar on Informatic Structures.
  \item (08 Jan 2024) Directions and Perspectives in the \(\lambda\)-calculus at the University of Bologna.
  \item (22 Nov 2023) 34th Nordic Workshop on Programming Theory.
  \item (01 Aug 2023) Applied Category Theory conference.
  \item (18 Jul 2023) Coresources workshop at the University of Cambridge.
  \item (27 Jun 2023) Logic in Computer Science conference.
  \item (25 Jun 2023) International Workshop on Quantitative Logical methods.
  \item (14 Jun 2023) Categories Networking Project opening workshop.
  \item (26 Apr 2023) Italian Category Theory Fest 2023.
  \item (01 Aug 2022) Women in Logic workshop.
  \item (22 Jul 2022) Applied Category Theory conference.
  \item (26 Jun 2022) 29th Foundational Methods in Computer Science workshop.
  \item (25 May 2022) Comonads Meetup at the University of Cambridge.
  \item (24 May 2022) Mathematical Foundations Seminar at the University of Bath.
  \item (13 Dec 2021) Symposium on Compositional Structures 8.
  \item (26 Jan 2021) Computer Science Logic 2021 conference.
\end{itemize}

%%%%%%%%%%%%%%%%%%%%%%%%%%%%%%
\resheading{Other skills}
%%%%%%%%%%%%%%%%%%%%%%%%%%%%%%

\begin{itemize}
  \item[] \textbf{Language skills:} Italian (native), English (C1).
  \item[] \textbf{Programming languages:} basic knowledge of Idris, Matlab, C, R.
\end{itemize}

%%%%%%%%%%%%%%%%%%%%%%%%%%%%%%
\resheading{Publications}
%%%%%%%%%%%%%%%%%%%%%%%%%%%%%%
\printbibliography[heading=none]
% \textbf{Note.} As customary in mathematics, all my publications list the authors in alphabetical order.

\end{document}
